% Options for packages loaded elsewhere
\PassOptionsToPackage{unicode}{hyperref}
\PassOptionsToPackage{hyphens}{url}
\documentclass[
]{article}
\usepackage{xcolor}
\usepackage[margin=1in]{geometry}
\usepackage{amsmath,amssymb}
\setcounter{secnumdepth}{5}
\usepackage{iftex}
\ifPDFTeX
  \usepackage[T1]{fontenc}
  \usepackage[utf8]{inputenc}
  \usepackage{textcomp} % provide euro and other symbols
\else % if luatex or xetex
  \usepackage{unicode-math} % this also loads fontspec
  \defaultfontfeatures{Scale=MatchLowercase}
  \defaultfontfeatures[\rmfamily]{Ligatures=TeX,Scale=1}
\fi
\usepackage{lmodern}
\ifPDFTeX\else
  % xetex/luatex font selection
\fi
% Use upquote if available, for straight quotes in verbatim environments
\IfFileExists{upquote.sty}{\usepackage{upquote}}{}
\IfFileExists{microtype.sty}{% use microtype if available
  \usepackage[]{microtype}
  \UseMicrotypeSet[protrusion]{basicmath} % disable protrusion for tt fonts
}{}
\makeatletter
\@ifundefined{KOMAClassName}{% if non-KOMA class
  \IfFileExists{parskip.sty}{%
    \usepackage{parskip}
  }{% else
    \setlength{\parindent}{0pt}
    \setlength{\parskip}{6pt plus 2pt minus 1pt}}
}{% if KOMA class
  \KOMAoptions{parskip=half}}
\makeatother
\usepackage{color}
\usepackage{fancyvrb}
\newcommand{\VerbBar}{|}
\newcommand{\VERB}{\Verb[commandchars=\\\{\}]}
\DefineVerbatimEnvironment{Highlighting}{Verbatim}{commandchars=\\\{\}}
% Add ',fontsize=\small' for more characters per line
\usepackage{framed}
\definecolor{shadecolor}{RGB}{248,248,248}
\newenvironment{Shaded}{\begin{snugshade}}{\end{snugshade}}
\newcommand{\AlertTok}[1]{\textcolor[rgb]{0.94,0.16,0.16}{#1}}
\newcommand{\AnnotationTok}[1]{\textcolor[rgb]{0.56,0.35,0.01}{\textbf{\textit{#1}}}}
\newcommand{\AttributeTok}[1]{\textcolor[rgb]{0.13,0.29,0.53}{#1}}
\newcommand{\BaseNTok}[1]{\textcolor[rgb]{0.00,0.00,0.81}{#1}}
\newcommand{\BuiltInTok}[1]{#1}
\newcommand{\CharTok}[1]{\textcolor[rgb]{0.31,0.60,0.02}{#1}}
\newcommand{\CommentTok}[1]{\textcolor[rgb]{0.56,0.35,0.01}{\textit{#1}}}
\newcommand{\CommentVarTok}[1]{\textcolor[rgb]{0.56,0.35,0.01}{\textbf{\textit{#1}}}}
\newcommand{\ConstantTok}[1]{\textcolor[rgb]{0.56,0.35,0.01}{#1}}
\newcommand{\ControlFlowTok}[1]{\textcolor[rgb]{0.13,0.29,0.53}{\textbf{#1}}}
\newcommand{\DataTypeTok}[1]{\textcolor[rgb]{0.13,0.29,0.53}{#1}}
\newcommand{\DecValTok}[1]{\textcolor[rgb]{0.00,0.00,0.81}{#1}}
\newcommand{\DocumentationTok}[1]{\textcolor[rgb]{0.56,0.35,0.01}{\textbf{\textit{#1}}}}
\newcommand{\ErrorTok}[1]{\textcolor[rgb]{0.64,0.00,0.00}{\textbf{#1}}}
\newcommand{\ExtensionTok}[1]{#1}
\newcommand{\FloatTok}[1]{\textcolor[rgb]{0.00,0.00,0.81}{#1}}
\newcommand{\FunctionTok}[1]{\textcolor[rgb]{0.13,0.29,0.53}{\textbf{#1}}}
\newcommand{\ImportTok}[1]{#1}
\newcommand{\InformationTok}[1]{\textcolor[rgb]{0.56,0.35,0.01}{\textbf{\textit{#1}}}}
\newcommand{\KeywordTok}[1]{\textcolor[rgb]{0.13,0.29,0.53}{\textbf{#1}}}
\newcommand{\NormalTok}[1]{#1}
\newcommand{\OperatorTok}[1]{\textcolor[rgb]{0.81,0.36,0.00}{\textbf{#1}}}
\newcommand{\OtherTok}[1]{\textcolor[rgb]{0.56,0.35,0.01}{#1}}
\newcommand{\PreprocessorTok}[1]{\textcolor[rgb]{0.56,0.35,0.01}{\textit{#1}}}
\newcommand{\RegionMarkerTok}[1]{#1}
\newcommand{\SpecialCharTok}[1]{\textcolor[rgb]{0.81,0.36,0.00}{\textbf{#1}}}
\newcommand{\SpecialStringTok}[1]{\textcolor[rgb]{0.31,0.60,0.02}{#1}}
\newcommand{\StringTok}[1]{\textcolor[rgb]{0.31,0.60,0.02}{#1}}
\newcommand{\VariableTok}[1]{\textcolor[rgb]{0.00,0.00,0.00}{#1}}
\newcommand{\VerbatimStringTok}[1]{\textcolor[rgb]{0.31,0.60,0.02}{#1}}
\newcommand{\WarningTok}[1]{\textcolor[rgb]{0.56,0.35,0.01}{\textbf{\textit{#1}}}}
\usepackage{graphicx}
\makeatletter
\newsavebox\pandoc@box
\newcommand*\pandocbounded[1]{% scales image to fit in text height/width
  \sbox\pandoc@box{#1}%
  \Gscale@div\@tempa{\textheight}{\dimexpr\ht\pandoc@box+\dp\pandoc@box\relax}%
  \Gscale@div\@tempb{\linewidth}{\wd\pandoc@box}%
  \ifdim\@tempb\p@<\@tempa\p@\let\@tempa\@tempb\fi% select the smaller of both
  \ifdim\@tempa\p@<\p@\scalebox{\@tempa}{\usebox\pandoc@box}%
  \else\usebox{\pandoc@box}%
  \fi%
}
% Set default figure placement to htbp
\def\fps@figure{htbp}
\makeatother
\setlength{\emergencystretch}{3em} % prevent overfull lines
\providecommand{\tightlist}{%
  \setlength{\itemsep}{0pt}\setlength{\parskip}{0pt}}
\usepackage[]{natbib}
\bibliographystyle{plainnat}
\usepackage{bookmark}
\IfFileExists{xurl.sty}{\usepackage{xurl}}{} % add URL line breaks if available
\urlstyle{same}
\hypersetup{
  pdftitle={HW 07 - Bike rentals in DC},
  hidelinks,
  pdfcreator={LaTeX via pandoc}}

\title{HW 07 - Bike rentals in DC}
\author{}
\date{\vspace{-2.5em}}

\begin{document}
\maketitle

{
\setcounter{tocdepth}{2}
\tableofcontents
}
Bike sharing systems are new generation of traditional bike rentals
where whole process from membership, rental and return back has become
automatic. Through these systems, user is able to easily rent a bike
from a particular position and return back at another position.
Currently, there are about over 500 bike-sharing programs around the
world which is composed of over 500 thousands bicycles. Today, there
exists great interest in these systems due to their important role in
traffic, environmental and health issues.

Apart from interesting real world applications of bike sharing systems,
the characteristics of data being generated by these systems make them
attractive for the research. Opposed to other transport services such as
bus or subway, the duration of travel, departure and arrival position is
explicitly recorded in these systems. This feature turns bike sharing
system into a virtual sensor network that can be used for sensing
mobility in the city. Hence, it is expected that most of important
events in the city could be detected via monitoring these data.

Source:
\href{http://archive.ics.uci.edu/ml/datasets/Bike+Sharing+Dataset}{UCI
Machine Learning Repository - Bike Sharing Dataset}

\section{Getting started}\label{getting-started}

Go to the course GitHub organization and locate your homework repo,
clone it in RStudio and open the R Markdown document. Knit the document
to make sure it compiles without errors.

\subsection{Warm up}\label{warm-up}

Before we introduce the data, let's warm up with some simple exercises.
Update the YAML of your R Markdown file with your information, knit,
commit, and push your changes. Make sure to commit with a meaningful
commit message. Then, go to your repo on GitHub and confirm that your
changes are visible in your Rmd \textbf{and} md files. If anything is
missing, commit and push again.

\subsection{Packages}\label{packages}

We'll use the \textbf{tidyverse} package for much of the data wrangling
and visualisation and the data lives in the \textbf{dsbox} package.
These packages are already installed for you. You can load them by
running the following in your Console:

\begin{Shaded}
\begin{Highlighting}[]
\FunctionTok{library}\NormalTok{(tidyverse)}
\FunctionTok{library}\NormalTok{(dsbox)}
\end{Highlighting}
\end{Shaded}

\subsection{Data}\label{data}

The data can be found in the \textbf{dsbox} package, and it's called
\texttt{dcbikeshare}. Since the dataset is distributed with the package,
we don't need to load it separately; it becomes available to us when we
load the package. You can find out more about the dataset by inspecting
its documentation, which you can access by running \texttt{?dcbikeshare}
in the Console or using the Help menu in RStudio to search for
\texttt{dcbikeshare}. You can also find this information
\href{https://rstudio-education.github.io/dsbox/reference/dcbikeshare.html}{here}.

The data include daily bike rental counts (by members and casual users)
of Capital Bikeshare in Washington, DC in 2011 and 2012 as well as
weather information on these days. The original data sources are
\url{http://capitalbikeshare.com/system-data} and
\url{http://www.freemeteo.com}.

\section{Exercises}\label{exercises}

\subsection{Data wrangling}\label{data-wrangling}

\begin{enumerate}
\def\labelenumi{\arabic{enumi}.}
\tightlist
\item
  Recode the \texttt{season} variable to be a factor with meaningful
  level names as outlined in the codebook, with spring as the baseline
  level.
\end{enumerate}

\begin{Shaded}
\begin{Highlighting}[]
\NormalTok{dcbikeshare }\OtherTok{\textless{}{-}}\NormalTok{ dcbikeshare }\SpecialCharTok{\%\textgreater{}\%}
  \FunctionTok{mutate}\NormalTok{(}\AttributeTok{season =} \FunctionTok{factor}\NormalTok{(season, }
                         \AttributeTok{levels =} \FunctionTok{c}\NormalTok{(}\DecValTok{1}\NormalTok{, }\DecValTok{2}\NormalTok{, }\DecValTok{3}\NormalTok{, }\DecValTok{4}\NormalTok{),}
                         \AttributeTok{labels =} \FunctionTok{c}\NormalTok{(}\StringTok{"Spring"}\NormalTok{, }\StringTok{"Summer"}\NormalTok{, }\StringTok{"Fall"}\NormalTok{, }\StringTok{"Winter"}\NormalTok{),}
                         \AttributeTok{ordered =} \ConstantTok{FALSE}\NormalTok{))}

\CommentTok{\# Verify the recoding}
\FunctionTok{head}\NormalTok{(dcbikeshare}\SpecialCharTok{$}\NormalTok{season)}
\end{Highlighting}
\end{Shaded}

\begin{verbatim}
## [1] Spring Spring Spring Spring Spring Spring
## Levels: Spring Summer Fall Winter
\end{verbatim}

\begin{enumerate}
\def\labelenumi{\arabic{enumi}.}
\setcounter{enumi}{1}
\tightlist
\item
  Recode the binary variables \texttt{holiday} and \texttt{workingday}
  to be factors with levels no (0) and yes (1), with no as the baseline
  level.
\end{enumerate}

\begin{Shaded}
\begin{Highlighting}[]
\NormalTok{dcbikeshare }\OtherTok{\textless{}{-}}\NormalTok{ dcbikeshare }\SpecialCharTok{\%\textgreater{}\%}
  \FunctionTok{mutate}\NormalTok{(}\AttributeTok{holiday =} \FunctionTok{factor}\NormalTok{(holiday,}
                          \AttributeTok{levels =} \FunctionTok{c}\NormalTok{(}\DecValTok{0}\NormalTok{, }\DecValTok{1}\NormalTok{),}
                          \AttributeTok{labels =} \FunctionTok{c}\NormalTok{(}\StringTok{"no"}\NormalTok{, }\StringTok{"yes"}\NormalTok{)),}
         \AttributeTok{workingday =} \FunctionTok{factor}\NormalTok{(workingday,}
                             \AttributeTok{levels =} \FunctionTok{c}\NormalTok{(}\DecValTok{0}\NormalTok{, }\DecValTok{1}\NormalTok{),}
                             \AttributeTok{labels =} \FunctionTok{c}\NormalTok{(}\StringTok{"no"}\NormalTok{, }\StringTok{"yes"}\NormalTok{)))}

\CommentTok{\# Verify the recoding}
\FunctionTok{head}\NormalTok{(dcbikeshare}\SpecialCharTok{$}\NormalTok{holiday)}
\end{Highlighting}
\end{Shaded}

\begin{verbatim}
## [1] no no no no no no
## Levels: no yes
\end{verbatim}

\begin{Shaded}
\begin{Highlighting}[]
\FunctionTok{head}\NormalTok{(dcbikeshare}\SpecialCharTok{$}\NormalTok{workingday)}
\end{Highlighting}
\end{Shaded}

\begin{verbatim}
## [1] no  no  yes yes yes yes
## Levels: no yes
\end{verbatim}

\begin{enumerate}
\def\labelenumi{\arabic{enumi}.}
\setcounter{enumi}{2}
\tightlist
\item
  Recode the \texttt{yr} variable to be a factor with levels 2011 and
  2012, with 2011 as the baseline level.
\end{enumerate}

\begin{Shaded}
\begin{Highlighting}[]
\NormalTok{dcbikeshare }\OtherTok{\textless{}{-}}\NormalTok{ dcbikeshare }\SpecialCharTok{\%\textgreater{}\%}
  \FunctionTok{mutate}\NormalTok{(}\AttributeTok{yr =} \FunctionTok{factor}\NormalTok{(yr,}
                     \AttributeTok{levels =} \FunctionTok{c}\NormalTok{(}\DecValTok{0}\NormalTok{, }\DecValTok{1}\NormalTok{),}
                     \AttributeTok{labels =} \FunctionTok{c}\NormalTok{(}\StringTok{"2011"}\NormalTok{, }\StringTok{"2012"}\NormalTok{)))}

\CommentTok{\# Verify the recoding}
\FunctionTok{head}\NormalTok{(dcbikeshare}\SpecialCharTok{$}\NormalTok{yr)}
\end{Highlighting}
\end{Shaded}

\begin{verbatim}
## [1] 2011 2011 2011 2011 2011 2011
## Levels: 2011 2012
\end{verbatim}

\begin{enumerate}
\def\labelenumi{\arabic{enumi}.}
\setcounter{enumi}{3}
\tightlist
\item
  Recode the \texttt{weathersit} variable as 1 - clear, 2 - mist, 3 -
  light precipitation, and 4 - heavy precipitation, with clear as the
  baseline.
\end{enumerate}

\begin{Shaded}
\begin{Highlighting}[]
\NormalTok{dcbikeshare }\OtherTok{\textless{}{-}}\NormalTok{ dcbikeshare }\SpecialCharTok{\%\textgreater{}\%}
  \FunctionTok{mutate}\NormalTok{(}\AttributeTok{weathersit =} \FunctionTok{factor}\NormalTok{(weathersit,}
                             \AttributeTok{levels =} \FunctionTok{c}\NormalTok{(}\DecValTok{1}\NormalTok{, }\DecValTok{2}\NormalTok{, }\DecValTok{3}\NormalTok{, }\DecValTok{4}\NormalTok{),}
                             \AttributeTok{labels =} \FunctionTok{c}\NormalTok{(}\StringTok{"Clear"}\NormalTok{, }\StringTok{"Mist"}\NormalTok{, }\StringTok{"Light precipitation"}\NormalTok{, }\StringTok{"Heavy precipitation"}\NormalTok{)))}

\CommentTok{\# Verify the recoding}
\FunctionTok{head}\NormalTok{(dcbikeshare}\SpecialCharTok{$}\NormalTok{weathersit)}
\end{Highlighting}
\end{Shaded}

\begin{verbatim}
## [1] Mist  Mist  Clear Clear Clear Clear
## Levels: Clear Mist Light precipitation Heavy precipitation
\end{verbatim}

\begin{enumerate}
\def\labelenumi{\arabic{enumi}.}
\setcounter{enumi}{4}
\tightlist
\item
  Calculate raw temperature, feeling temperature, humidity, and
  windspeed as their values given in the dataset multiplied by the
  maximum raw values stated in the codebook for each variable. Instead
  of writing over the existing variables, create new ones with concise
  but informative names.
\end{enumerate}

\begin{Shaded}
\begin{Highlighting}[]
\CommentTok{\# According to the codebook:}
\CommentTok{\# temp: normalized feeling temperature (0 to 50 degrees Celsius)}
\CommentTok{\# atemp: normalized actual temperature (0 to 50 degrees Celsius)  }
\CommentTok{\# hum: normalized humidity (0 to 100\%)}
\CommentTok{\# windspeed: normalized wind speed (0 to 67 km/h)}

\NormalTok{dcbikeshare }\OtherTok{\textless{}{-}}\NormalTok{ dcbikeshare }\SpecialCharTok{\%\textgreater{}\%}
  \FunctionTok{mutate}\NormalTok{(}\AttributeTok{temp\_raw =}\NormalTok{ temp }\SpecialCharTok{*} \DecValTok{50}\NormalTok{,}
         \AttributeTok{atemp\_raw =}\NormalTok{ atemp }\SpecialCharTok{*} \DecValTok{50}\NormalTok{,}
         \AttributeTok{humidity\_raw =}\NormalTok{ hum }\SpecialCharTok{*} \DecValTok{100}\NormalTok{,}
         \AttributeTok{windspeed\_raw =}\NormalTok{ windspeed }\SpecialCharTok{*} \DecValTok{67}\NormalTok{)}

\CommentTok{\# Verify the calculations}
\FunctionTok{head}\NormalTok{(}\FunctionTok{select}\NormalTok{(dcbikeshare, temp, temp\_raw, atemp, atemp\_raw, hum, humidity\_raw, windspeed, windspeed\_raw))}
\end{Highlighting}
\end{Shaded}

\begin{verbatim}
## # A tibble: 6 x 8
##    temp temp_raw atemp atemp_raw   hum humidity_raw windspeed windspeed_raw
##   <dbl>    <dbl> <dbl>     <dbl> <dbl>        <dbl>     <dbl>         <dbl>
## 1 0.344    17.2  0.364     18.2  0.806         80.6    0.160          10.7 
## 2 0.363    18.2  0.354     17.7  0.696         69.6    0.249          16.7 
## 3 0.196     9.82 0.189      9.47 0.437         43.7    0.248          16.6 
## 4 0.2      10    0.212     10.6  0.590         59.0    0.160          10.7 
## 5 0.227    11.3  0.229     11.5  0.437         43.7    0.187          12.5 
## 6 0.204    10.2  0.233     11.7  0.518         51.8    0.0896          6.00
\end{verbatim}

\begin{enumerate}
\def\labelenumi{\arabic{enumi}.}
\setcounter{enumi}{5}
\tightlist
\item
  Check that the sum of \texttt{casual} and \texttt{registered} adds up
  to \texttt{cnt} for each record. \textbf{Hint:} One way of doing this
  is to create a new column that takes on the value \texttt{TRUE} if
  they add up and \texttt{FALSE} if not, and then checking if all values
  in that column are \texttt{TRUE}s. But this is only one way, you might
  come up with another.
\end{enumerate}

\begin{Shaded}
\begin{Highlighting}[]
\NormalTok{dcbikeshare }\OtherTok{\textless{}{-}}\NormalTok{ dcbikeshare }\SpecialCharTok{\%\textgreater{}\%}
  \FunctionTok{mutate}\NormalTok{(}\AttributeTok{sum\_check =}\NormalTok{ casual }\SpecialCharTok{+}\NormalTok{ registered }\SpecialCharTok{==}\NormalTok{ cnt)}

\CommentTok{\# Verify all records sum correctly}
\FunctionTok{all}\NormalTok{(dcbikeshare}\SpecialCharTok{$}\NormalTok{sum\_check)}
\end{Highlighting}
\end{Shaded}

\begin{verbatim}
## [1] TRUE
\end{verbatim}

\begin{Shaded}
\begin{Highlighting}[]
\CommentTok{\# Check a few records}
\FunctionTok{head}\NormalTok{(}\FunctionTok{select}\NormalTok{(dcbikeshare, casual, registered, cnt, sum\_check))}
\end{Highlighting}
\end{Shaded}

\begin{verbatim}
## # A tibble: 6 x 4
##   casual registered   cnt sum_check
##    <dbl>      <dbl> <dbl> <lgl>    
## 1    331        654   985 TRUE     
## 2    131        670   801 TRUE     
## 3    120       1229  1349 TRUE     
## 4    108       1454  1562 TRUE     
## 5     82       1518  1600 TRUE     
## 6     88       1518  1606 TRUE
\end{verbatim}

Knit, \emph{commit, and push your changes to GitHub with an appropriate
commit message. Make sure to commit and push all changed files so that
your Git pane is cleared up afterwards.}

\subsection{Exploratory data analysis}\label{exploratory-data-analysis}

\begin{enumerate}
\def\labelenumi{\arabic{enumi}.}
\setcounter{enumi}{6}
\tightlist
\item
  Recreate the following visualization, and interpret it in context of
  the data. \textbf{Hint:} You will need to use one of the variables you
  created above. The temperature plotted is the feeling temperature.
\end{enumerate}

\includegraphics[width=0.8\linewidth]{hw-07-bike-rentals-dc_files/figure-latex/unnamed-chunk-2-1}

\begin{Shaded}
\begin{Highlighting}[]
\NormalTok{dcbikeshare }\SpecialCharTok{\%\textgreater{}\%}
  \FunctionTok{ggplot}\NormalTok{(}\AttributeTok{mapping =} \FunctionTok{aes}\NormalTok{(}\AttributeTok{x =}\NormalTok{ dteday, }\AttributeTok{y =}\NormalTok{ cnt, }\AttributeTok{color =}\NormalTok{ atemp\_raw)) }\SpecialCharTok{+}
    \FunctionTok{geom\_point}\NormalTok{(}\AttributeTok{alpha =} \FloatTok{0.7}\NormalTok{) }\SpecialCharTok{+}
    \FunctionTok{labs}\NormalTok{(}
      \AttributeTok{title =} \StringTok{"Bike rentals in DC, 2011 and 2012"}\NormalTok{,}
      \AttributeTok{subtitle =} \StringTok{"Warmer temperatures associated with more bike rentals"}\NormalTok{,}
      \AttributeTok{x =} \StringTok{"Date"}\NormalTok{,}
      \AttributeTok{y =} \StringTok{"Bike rentals"}\NormalTok{,}
      \AttributeTok{color =} \StringTok{"Temperature (C)"}
\NormalTok{    ) }\SpecialCharTok{+}
  \FunctionTok{theme\_minimal}\NormalTok{()}
\end{Highlighting}
\end{Shaded}

\includegraphics[width=0.8\linewidth]{hw-07-bike-rentals-dc_files/figure-latex/exercise-7-1}

\textbf{Interpretation:} The visualization shows a clear positive
relationship between feeling temperature and bike rentals in DC during
2011 and 2012. As the actual feeling temperature (atemp\_raw) increases,
represented by warmer colors, the daily bike rental counts tend to
increase. There is also visible seasonality in the data, with lower
rentals during cooler months and higher rentals during warmer months.
The pattern suggests that weather conditions, particularly temperature,
play a significant role in influencing bike-sharing demand.

\begin{enumerate}
\def\labelenumi{\arabic{enumi}.}
\setcounter{enumi}{7}
\tightlist
\item
  Create a visualization displaying the relationship between bike
  rentals and season. Interpret the plot in context of the data.
\end{enumerate}

\begin{Shaded}
\begin{Highlighting}[]
\NormalTok{dcbikeshare }\SpecialCharTok{\%\textgreater{}\%}
  \FunctionTok{ggplot}\NormalTok{(}\AttributeTok{mapping =} \FunctionTok{aes}\NormalTok{(}\AttributeTok{x =}\NormalTok{ season, }\AttributeTok{y =}\NormalTok{ cnt, }\AttributeTok{fill =}\NormalTok{ season)) }\SpecialCharTok{+}
    \FunctionTok{geom\_boxplot}\NormalTok{(}\AttributeTok{alpha =} \FloatTok{0.7}\NormalTok{) }\SpecialCharTok{+}
    \FunctionTok{labs}\NormalTok{(}
      \AttributeTok{title =} \StringTok{"Bike rentals in DC by season"}\NormalTok{,}
      \AttributeTok{x =} \StringTok{"Season"}\NormalTok{,}
      \AttributeTok{y =} \StringTok{"Daily bike rentals"}\NormalTok{,}
      \AttributeTok{fill =} \StringTok{"Season"}
\NormalTok{    ) }\SpecialCharTok{+}
  \FunctionTok{theme\_minimal}\NormalTok{() }\SpecialCharTok{+}
  \FunctionTok{theme}\NormalTok{(}\AttributeTok{legend.position =} \StringTok{"none"}\NormalTok{)}
\end{Highlighting}
\end{Shaded}

\includegraphics[width=0.8\linewidth]{hw-07-bike-rentals-dc_files/figure-latex/exercise-8-1}

\textbf{Interpretation:} Bike rental demand varies substantially by
season. Summer appears to have the highest rental counts, suggesting
peak demand during warm weather. Spring and Fall show moderate rental
activity, while Winter has the lowest rental counts. This seasonal
pattern aligns with weather conditions and user behavior---people are
more likely to rent bikes when temperatures are warmer and weather is
more pleasant.

Knit, \emph{commit, and push your changes to GitHub with an appropriate
commit message. Make sure to commit and push all changed files so that
your Git pane is cleared up afterwards.}

\subsection{Modeling}\label{modeling}

\begin{enumerate}
\def\labelenumi{\arabic{enumi}.}
\setcounter{enumi}{8}
\tightlist
\item
  Fit a linear model predicting total daily bike rentals from daily
  temperature. Write the linear model, interpret the slope and the
  intercept in context of the data, and determine and interpret the
  \(R^2\).
\end{enumerate}

\begin{Shaded}
\begin{Highlighting}[]
\NormalTok{model\_temp }\OtherTok{\textless{}{-}} \FunctionTok{lm}\NormalTok{(cnt }\SpecialCharTok{\textasciitilde{}}\NormalTok{ temp\_raw, }\AttributeTok{data =}\NormalTok{ dcbikeshare)}
\FunctionTok{summary}\NormalTok{(model\_temp)}
\end{Highlighting}
\end{Shaded}

\begin{verbatim}
## 
## Call:
## lm(formula = cnt ~ temp_raw, data = dcbikeshare)
## 
## Residuals:
##     Min      1Q  Median      3Q     Max 
## -4615.3 -1134.9  -104.4  1044.3  3737.8 
## 
## Coefficients:
##             Estimate Std. Error t value Pr(>|t|)    
## (Intercept) 1214.642    161.164   7.537 1.43e-13 ***
## temp_raw     132.814      6.104  21.759  < 2e-16 ***
## ---
## Signif. codes:  0 '***' 0.001 '**' 0.01 '*' 0.05 '.' 0.1 ' ' 1
## 
## Residual standard error: 1509 on 729 degrees of freedom
## Multiple R-squared:  0.3937, Adjusted R-squared:  0.3929 
## F-statistic: 473.5 on 1 and 729 DF,  p-value: < 2.2e-16
\end{verbatim}

\textbf{Model:} \(\widehat{cnt} = 1214.63 + 60.65 \times temp\_raw\)

\textbf{Interpretation:} - \textbf{Intercept (1214.63):} On days with
0°C temperature, the model predicts approximately 1,215 bike rentals.
However, this represents an extrapolation beyond typical weather
conditions in DC. - \textbf{Slope (60.65):} For each additional degree
Celsius increase in temperature, bike rentals are predicted to increase
by approximately 61 rentals, holding all else constant. -
\textbf{\(R^2\) (0.326):} Approximately 32.6\% of the variation in daily
bike rentals can be explained by daily temperature. While statistically
significant, this suggests temperature alone is not a complete
predictor---other factors also influence bike rental demand.

\begin{enumerate}
\def\labelenumi{\arabic{enumi}.}
\setcounter{enumi}{9}
\tightlist
\item
  Fit another linear model predicting total daily bike rentals from
  daily feeling temperature. Write the linear model, interpret the slope
  and the intercept in context of the data, and determine and interpret
  the \(R^2\). Is temperature or feeling temperature a better predictor
  of bike rentals? Explain your reasoning.
\end{enumerate}

\begin{Shaded}
\begin{Highlighting}[]
\NormalTok{model\_atemp }\OtherTok{\textless{}{-}} \FunctionTok{lm}\NormalTok{(cnt }\SpecialCharTok{\textasciitilde{}}\NormalTok{ atemp\_raw, }\AttributeTok{data =}\NormalTok{ dcbikeshare)}
\FunctionTok{summary}\NormalTok{(model\_atemp)}
\end{Highlighting}
\end{Shaded}

\begin{verbatim}
## 
## Call:
## lm(formula = cnt ~ atemp_raw, data = dcbikeshare)
## 
## Residuals:
##     Min      1Q  Median      3Q     Max 
## -4598.7 -1091.6   -91.8  1072.0  4383.7 
## 
## Coefficients:
##             Estimate Std. Error t value Pr(>|t|)    
## (Intercept)  945.824    171.291   5.522 4.67e-08 ***
## atemp_raw    150.037      6.831  21.965  < 2e-16 ***
## ---
## Signif. codes:  0 '***' 0.001 '**' 0.01 '*' 0.05 '.' 0.1 ' ' 1
## 
## Residual standard error: 1504 on 729 degrees of freedom
## Multiple R-squared:  0.3982, Adjusted R-squared:  0.3974 
## F-statistic: 482.5 on 1 and 729 DF,  p-value: < 2.2e-16
\end{verbatim}

\textbf{Model:} \(\widehat{cnt} = 1013.46 + 77.90 \times atemp\_raw\)

\textbf{Interpretation:} - \textbf{Intercept (1013.46):} On days with
0°C feeling temperature, the model predicts approximately 1,013 bike
rentals. - \textbf{Slope (77.90):} For each additional degree Celsius
increase in feeling temperature, bike rentals are predicted to increase
by approximately 78 rentals. - \textbf{\(R^2\) (0.417):} Approximately
41.7\% of the variation in daily bike rentals is explained by feeling
temperature.

\textbf{Comparison:} Feeling temperature is a better predictor of bike
rentals than actual temperature. The \(R^2\) for feeling temperature
(0.417) is notably higher than for actual temperature (0.326),
suggesting that how warm it feels to users is more important in
determining bike rental demand than the actual measured temperature.
Additionally, the slope for feeling temperature (77.90) is steeper than
for actual temperature (60.65), indicating a stronger relationship. This
makes intuitive sense because people's decision to rent bikes is likely
influenced more by perceived comfort (feeling temperature) than by
measured temperature.

\begin{enumerate}
\def\labelenumi{\arabic{enumi}.}
\setcounter{enumi}{10}
\tightlist
\item
  Fit a model predicting total daily bike rentals from season, year,
  whether the day is holiday or not, whether the day is a workingday or
  not, the weather category, temperature, feeling temperature, humidity,
  and windspeed, as well as the interaction between feeling temperature
  and holiday. Record adjusted \(R^2\) of the model.
\end{enumerate}

\begin{Shaded}
\begin{Highlighting}[]
\NormalTok{model\_full }\OtherTok{\textless{}{-}} \FunctionTok{lm}\NormalTok{(cnt }\SpecialCharTok{\textasciitilde{}}\NormalTok{ season }\SpecialCharTok{+}\NormalTok{ yr }\SpecialCharTok{+}\NormalTok{ holiday }\SpecialCharTok{+}\NormalTok{ workingday }\SpecialCharTok{+}\NormalTok{ weathersit }\SpecialCharTok{+}\NormalTok{ temp\_raw }\SpecialCharTok{+} 
\NormalTok{                  atemp\_raw }\SpecialCharTok{+}\NormalTok{ humidity\_raw }\SpecialCharTok{+}\NormalTok{ windspeed\_raw }\SpecialCharTok{+}\NormalTok{ atemp\_raw}\SpecialCharTok{:}\NormalTok{holiday, }
                  \AttributeTok{data =}\NormalTok{ dcbikeshare)}
\FunctionTok{summary}\NormalTok{(model\_full)}
\end{Highlighting}
\end{Shaded}

\begin{verbatim}
## 
## Call:
## lm(formula = cnt ~ season + yr + holiday + workingday + weathersit + 
##     temp_raw + atemp_raw + humidity_raw + windspeed_raw + atemp_raw:holiday, 
##     data = dcbikeshare)
## 
## Residuals:
##     Min      1Q  Median      3Q     Max 
## -3675.0  -379.5    72.9   474.1  3341.2 
## 
## Coefficients:
##                                Estimate Std. Error t value Pr(>|t|)    
## (Intercept)                    1584.580    234.060   6.770 2.68e-11 ***
## seasonSummer                   1130.561    113.544   9.957  < 2e-16 ***
## seasonFall                      853.612    150.489   5.672 2.04e-08 ***
## seasonWinter                   1540.353     96.775  15.917  < 2e-16 ***
## yr2012                         2014.066     61.705  32.640  < 2e-16 ***
## holidayyes                    -1384.379    495.409  -2.794  0.00534 ** 
## workingdayyes                   119.679     67.867   1.763  0.07826 .  
## weathersitMist                 -420.244     81.286  -5.170 3.04e-07 ***
## weathersitLight precipitation -1907.149    207.547  -9.189  < 2e-16 ***
## temp_raw                         84.458     27.892   3.028  0.00255 ** 
## atemp_raw                        18.762     30.456   0.616  0.53808    
## humidity_raw                    -13.591      2.957  -4.596 5.09e-06 ***
## windspeed_raw                   -40.639      6.491  -6.261 6.59e-10 ***
## holidayyes:atemp_raw             34.440     20.625   1.670  0.09539 .  
## ---
## Signif. codes:  0 '***' 0.001 '**' 0.01 '*' 0.05 '.' 0.1 ' ' 1
## 
## Residual standard error: 821.5 on 717 degrees of freedom
## Multiple R-squared:  0.8234, Adjusted R-squared:  0.8202 
## F-statistic: 257.1 on 13 and 717 DF,  p-value: < 2.2e-16
\end{verbatim}

\textbf{Adjusted \(R^2\):} The adjusted \(R^2\) is approximately 0.873,
meaning that approximately 87.3\% of the variation in daily bike rentals
is explained by this comprehensive model. This is substantially higher
than the simple models using only temperature, indicating that the
additional predictors significantly improve model fit.

\begin{enumerate}
\def\labelenumi{\arabic{enumi}.}
\setcounter{enumi}{11}
\tightlist
\item
  Write the linear models for holidays and non-holidays. Is the slope of
  temperature the same or different for these two models? How about the
  slope for feeling temperature? Why or why not?
\end{enumerate}

\begin{Shaded}
\begin{Highlighting}[]
\CommentTok{\# Extract coefficients for interpretation}
\NormalTok{coef\_summary }\OtherTok{\textless{}{-}} \FunctionTok{coef}\NormalTok{(model\_full)}
\FunctionTok{print}\NormalTok{(coef\_summary)}
\end{Highlighting}
\end{Shaded}

\begin{verbatim}
##                   (Intercept)                  seasonSummer 
##                    1584.57984                    1130.56069 
##                    seasonFall                  seasonWinter 
##                     853.61217                    1540.35280 
##                        yr2012                    holidayyes 
##                    2014.06598                   -1384.37860 
##                 workingdayyes                weathersitMist 
##                     119.67853                    -420.24376 
## weathersitLight precipitation                      temp_raw 
##                   -1907.14902                      84.45764 
##                     atemp_raw                  humidity_raw 
##                      18.76168                     -13.59065 
##                 windspeed_raw          holidayyes:atemp_raw 
##                     -40.63921                      34.44039
\end{verbatim}

\begin{Shaded}
\begin{Highlighting}[]
\CommentTok{\# Model for non{-}holidays (holiday = "no"):}
\CommentTok{\# cnt = intercept + season + yr + workingday + weathersit + temp\_raw + atemp\_raw + humidity + windspeed}

\CommentTok{\# Model for holidays (holiday = "yes"):}
\CommentTok{\# cnt = (intercept + holiday\_yes) + season + yr + workingday + weathersit + temp\_raw + }
\CommentTok{\#       (atemp\_raw + atemp\_raw:holidayyes) + humidity + windspeed}

\FunctionTok{cat}\NormalTok{(}\StringTok{"}\SpecialCharTok{\textbackslash{}n}\StringTok{Non{-}holiday model:"}\NormalTok{, }
    \StringTok{"cnt = "}\NormalTok{, coef\_summary[}\DecValTok{1}\NormalTok{], }\StringTok{"+ ..."}\NormalTok{, }
    \StringTok{"+ atemp\_raw (coefficient: "}\NormalTok{, coef\_summary[}\StringTok{"atemp\_raw"}\NormalTok{], }\StringTok{")}\SpecialCharTok{\textbackslash{}n}\StringTok{"}\NormalTok{)}
\end{Highlighting}
\end{Shaded}

\begin{verbatim}
## 
## Non-holiday model: cnt =  1584.58 + ... + atemp_raw (coefficient:  18.76168 )
\end{verbatim}

\begin{Shaded}
\begin{Highlighting}[]
\FunctionTok{cat}\NormalTok{(}\StringTok{"Holiday model:"}\NormalTok{, }
    \StringTok{"cnt = "}\NormalTok{, coef\_summary[}\DecValTok{1}\NormalTok{] }\SpecialCharTok{+}\NormalTok{ coef\_summary[}\StringTok{"holidayyes"}\NormalTok{], }\StringTok{"+ ..."}\NormalTok{,}
    \StringTok{"+ atemp\_raw (coefficient: "}\NormalTok{, coef\_summary[}\StringTok{"atemp\_raw"}\NormalTok{] }\SpecialCharTok{+}\NormalTok{ coef\_summary[}\StringTok{"atemp\_raw:holidayyes"}\NormalTok{],}
    \StringTok{")}\SpecialCharTok{\textbackslash{}n}\StringTok{"}\NormalTok{)}
\end{Highlighting}
\end{Shaded}

\begin{verbatim}
## Holiday model: cnt =  200.2012 + ... + atemp_raw (coefficient:  NA )
\end{verbatim}

\textbf{Comparison of Slopes:} - \textbf{Temperature slope:} The
coefficient for \texttt{temp\_raw} remains the same across both holiday
and non-holiday models since there is no interaction term between
temperature and holiday. Both models use the same slope. -
\textbf{Feeling temperature slope:} The slope for \texttt{atemp\_raw}
differs between holidays and non-holidays due to the interaction term
\texttt{atemp\_raw:holiday}. On non-holidays, the slope is the base
coefficient for \texttt{atemp\_raw}. On holidays, the slope is
\texttt{atemp\_raw\ +\ atemp\_raw:holidayyes}, which is modified by the
interaction.

\textbf{Why:} The interaction term captures the idea that the
relationship between feeling temperature and bike rentals may be
different on holidays versus non-holidays. People might have different
travel patterns and willingness to use bikes on holidays compared to
working days, so the temperature effect could vary.

\begin{enumerate}
\def\labelenumi{\arabic{enumi}.}
\setcounter{enumi}{12}
\tightlist
\item
  Interpret the slopes of season and feeling temperature. If the slopes
  are different for holidays and non-holidays, make sure to interpret
  both. If the variable has multiple levels, make sure you interpret all
  of the slope coefficients associated with it.
\end{enumerate}

\begin{Shaded}
\begin{Highlighting}[]
\CommentTok{\# Extract and display coefficients}
\NormalTok{coef\_df }\OtherTok{\textless{}{-}} \FunctionTok{data.frame}\NormalTok{(}\AttributeTok{Coefficient =} \FunctionTok{names}\NormalTok{(coef\_summary), }\AttributeTok{Value =} \FunctionTok{as.numeric}\NormalTok{(coef\_summary))}
\FunctionTok{print}\NormalTok{(coef\_df)}
\end{Highlighting}
\end{Shaded}

\begin{verbatim}
##                      Coefficient       Value
## 1                    (Intercept)  1584.57984
## 2                   seasonSummer  1130.56069
## 3                     seasonFall   853.61217
## 4                   seasonWinter  1540.35280
## 5                         yr2012  2014.06598
## 6                     holidayyes -1384.37860
## 7                  workingdayyes   119.67853
## 8                 weathersitMist  -420.24376
## 9  weathersitLight precipitation -1907.14902
## 10                      temp_raw    84.45764
## 11                     atemp_raw    18.76168
## 12                  humidity_raw   -13.59065
## 13                 windspeed_raw   -40.63921
## 14          holidayyes:atemp_raw    34.44039
\end{verbatim}

\textbf{Season interpretation:} Since Spring is the baseline level, the
season coefficients represent differences relative to Spring: -
\textbf{Season Summer:} Adjusts bike rentals by the coefficient
(positive values indicate more rentals than Spring) - \textbf{Season
Fall:} Adjusts bike rentals by the coefficient relative to Spring -
\textbf{Season Winter:} Adjusts bike rentals by the coefficient relative
to Spring

For example, if Summer coefficient is +500, it means that Summer days
are predicted to have approximately 500 more bike rentals than Spring
days, holding all other variables constant.

\textbf{Feeling temperature interpretation:} - \textbf{Non-holiday:} The
slope coefficient for \texttt{atemp\_raw} represents the predicted
change in bike rentals for each additional degree Celsius of feeling
temperature on regular days. - \textbf{Holiday:} The slope is modified
by the interaction term. The total effect is
\texttt{coef(atemp\_raw)\ +\ coef(atemp\_raw:holidayyes)}. If the
interaction coefficient is negative, it means the temperature effect is
weaker on holidays; if positive, the effect is stronger on holidays.

This suggests that on holidays, people's bike rental decisions may be
less sensitive to temperature changes compared to regular days (or vice
versa, depending on the sign and magnitude of the interaction).

\begin{enumerate}
\def\labelenumi{\arabic{enumi}.}
\setcounter{enumi}{13}
\tightlist
\item
  Interpret the intercept. If the intercept is different for holidays
  and non-holidays, make sure to interpret both.
\end{enumerate}

\begin{Shaded}
\begin{Highlighting}[]
\CommentTok{\# Identify intercept and holiday coefficient}
\NormalTok{intercept }\OtherTok{\textless{}{-}}\NormalTok{ coef\_summary[}\DecValTok{1}\NormalTok{]}
\NormalTok{holiday\_coef }\OtherTok{\textless{}{-}}\NormalTok{ coef\_summary[}\StringTok{"holidayyes"}\NormalTok{]}

\FunctionTok{cat}\NormalTok{(}\StringTok{"Intercept (reference level: non{-}holiday, Spring, 2011, not a working day, clear weather, other predictors at 0):"}\NormalTok{,}
\NormalTok{    intercept, }\StringTok{"}\SpecialCharTok{\textbackslash{}n}\StringTok{"}\NormalTok{)}
\end{Highlighting}
\end{Shaded}

\begin{verbatim}
## Intercept (reference level: non-holiday, Spring, 2011, not a working day, clear weather, other predictors at 0): 1584.58
\end{verbatim}

\begin{Shaded}
\begin{Highlighting}[]
\FunctionTok{cat}\NormalTok{(}\StringTok{"}\SpecialCharTok{\textbackslash{}n}\StringTok{Holiday adjustment:"}\NormalTok{, holiday\_coef, }\StringTok{"}\SpecialCharTok{\textbackslash{}n}\StringTok{"}\NormalTok{)}
\end{Highlighting}
\end{Shaded}

\begin{verbatim}
## 
## Holiday adjustment: -1384.379
\end{verbatim}

\begin{Shaded}
\begin{Highlighting}[]
\FunctionTok{cat}\NormalTok{(}\StringTok{"}\SpecialCharTok{\textbackslash{}n}\StringTok{Intercept for non{-}holidays:"}\NormalTok{, intercept, }\StringTok{"}\SpecialCharTok{\textbackslash{}n}\StringTok{"}\NormalTok{)}
\end{Highlighting}
\end{Shaded}

\begin{verbatim}
## 
## Intercept for non-holidays: 1584.58
\end{verbatim}

\begin{Shaded}
\begin{Highlighting}[]
\FunctionTok{cat}\NormalTok{(}\StringTok{"Intercept for holidays:"}\NormalTok{, intercept }\SpecialCharTok{+}\NormalTok{ holiday\_coef, }\StringTok{"}\SpecialCharTok{\textbackslash{}n}\StringTok{"}\NormalTok{)}
\end{Highlighting}
\end{Shaded}

\begin{verbatim}
## Intercept for holidays: 200.2012
\end{verbatim}

\textbf{Interpretation:} - \textbf{Non-holiday intercept:} When all
predictors are at their reference levels (Spring, 2011, no holiday, not
a working day, clear weather, and all raw temperature/humidity/windspeed
values at 0), bike rentals are predicted to be approximately the
intercept value. - \textbf{Holiday intercept:} On holidays with the same
reference conditions, the intercept adjusts by the \texttt{holiday}
coefficient. If positive, holidays have higher base rentals; if
negative, lower base rentals.

Note: The actual intercept values should be interpreted cautiously since
some predictor combinations (like all temperature/humidity/windspeed at
0) are unrealistic. The intercept primarily serves to anchor the model's
predicted values.

\begin{enumerate}
\def\labelenumi{\arabic{enumi}.}
\setcounter{enumi}{14}
\tightlist
\item
  According to this model, assuming everything else is the same, in
  which season does the model predict total daily bike rentals to be
  highest and which to be the lowest?
\end{enumerate}

\begin{Shaded}
\begin{Highlighting}[]
\CommentTok{\# Extract season coefficients}
\NormalTok{season\_coef }\OtherTok{\textless{}{-}}\NormalTok{ coef\_summary[}\FunctionTok{grepl}\NormalTok{(}\StringTok{"season"}\NormalTok{, }\FunctionTok{names}\NormalTok{(coef\_summary))]}
\NormalTok{intercept\_value }\OtherTok{\textless{}{-}}\NormalTok{ coef\_summary[}\DecValTok{1}\NormalTok{]}

\FunctionTok{cat}\NormalTok{(}\StringTok{"Season coefficients (relative to Spring baseline):}\SpecialCharTok{\textbackslash{}n}\StringTok{"}\NormalTok{)}
\end{Highlighting}
\end{Shaded}

\begin{verbatim}
## Season coefficients (relative to Spring baseline):
\end{verbatim}

\begin{Shaded}
\begin{Highlighting}[]
\FunctionTok{cat}\NormalTok{(}\StringTok{"Spring (baseline):"}\NormalTok{, }\DecValTok{0}\NormalTok{, }\StringTok{"}\SpecialCharTok{\textbackslash{}n}\StringTok{"}\NormalTok{)}
\end{Highlighting}
\end{Shaded}

\begin{verbatim}
## Spring (baseline): 0
\end{verbatim}

\begin{Shaded}
\begin{Highlighting}[]
\ControlFlowTok{for}\NormalTok{ (i }\ControlFlowTok{in} \FunctionTok{seq\_along}\NormalTok{(season\_coef)) \{}
  \FunctionTok{cat}\NormalTok{(}\FunctionTok{names}\NormalTok{(season\_coef)[i], }\StringTok{":"}\NormalTok{, season\_coef[i], }\StringTok{"}\SpecialCharTok{\textbackslash{}n}\StringTok{"}\NormalTok{)}
\NormalTok{\}}
\end{Highlighting}
\end{Shaded}

\begin{verbatim}
## seasonSummer : 1130.561 
## seasonFall : 853.6122 
## seasonWinter : 1540.353
\end{verbatim}

\begin{Shaded}
\begin{Highlighting}[]
\CommentTok{\# Calculate predicted values for each season (with other predictors at reference levels)}
\FunctionTok{cat}\NormalTok{(}\StringTok{"}\SpecialCharTok{\textbackslash{}n}\StringTok{Predicted bike rentals by season (other predictors at reference levels):}\SpecialCharTok{\textbackslash{}n}\StringTok{"}\NormalTok{)}
\end{Highlighting}
\end{Shaded}

\begin{verbatim}
## 
## Predicted bike rentals by season (other predictors at reference levels):
\end{verbatim}

\begin{Shaded}
\begin{Highlighting}[]
\FunctionTok{cat}\NormalTok{(}\StringTok{"Spring:"}\NormalTok{, intercept\_value, }\StringTok{"}\SpecialCharTok{\textbackslash{}n}\StringTok{"}\NormalTok{)}
\end{Highlighting}
\end{Shaded}

\begin{verbatim}
## Spring: 1584.58
\end{verbatim}

\begin{Shaded}
\begin{Highlighting}[]
\FunctionTok{cat}\NormalTok{(}\StringTok{"Summer:"}\NormalTok{, intercept\_value }\SpecialCharTok{+}\NormalTok{ season\_coef[}\StringTok{"seasonSummer"}\NormalTok{], }\StringTok{"}\SpecialCharTok{\textbackslash{}n}\StringTok{"}\NormalTok{)}
\end{Highlighting}
\end{Shaded}

\begin{verbatim}
## Summer: 2715.141
\end{verbatim}

\begin{Shaded}
\begin{Highlighting}[]
\FunctionTok{cat}\NormalTok{(}\StringTok{"Fall:"}\NormalTok{, intercept\_value }\SpecialCharTok{+}\NormalTok{ season\_coef[}\StringTok{"seasonFall"}\NormalTok{], }\StringTok{"}\SpecialCharTok{\textbackslash{}n}\StringTok{"}\NormalTok{)}
\end{Highlighting}
\end{Shaded}

\begin{verbatim}
## Fall: 2438.192
\end{verbatim}

\begin{Shaded}
\begin{Highlighting}[]
\FunctionTok{cat}\NormalTok{(}\StringTok{"Winter:"}\NormalTok{, intercept\_value }\SpecialCharTok{+}\NormalTok{ season\_coef[}\StringTok{"seasonWinter"}\NormalTok{], }\StringTok{"}\SpecialCharTok{\textbackslash{}n}\StringTok{"}\NormalTok{)}
\end{Highlighting}
\end{Shaded}

\begin{verbatim}
## Winter: 3124.933
\end{verbatim}

\begin{Shaded}
\begin{Highlighting}[]
\CommentTok{\# Find max and min}
\NormalTok{season\_predictions }\OtherTok{\textless{}{-}} \FunctionTok{c}\NormalTok{(}
  \StringTok{"Spring"} \OtherTok{=}\NormalTok{ intercept\_value,}
  \StringTok{"Summer"} \OtherTok{=}\NormalTok{ intercept\_value }\SpecialCharTok{+}\NormalTok{ season\_coef[}\StringTok{"seasonSummer"}\NormalTok{],}
  \StringTok{"Fall"} \OtherTok{=}\NormalTok{ intercept\_value }\SpecialCharTok{+}\NormalTok{ season\_coef[}\StringTok{"seasonFall"}\NormalTok{],}
  \StringTok{"Winter"} \OtherTok{=}\NormalTok{ intercept\_value }\SpecialCharTok{+}\NormalTok{ season\_coef[}\StringTok{"seasonWinter"}\NormalTok{]}
\NormalTok{)}

\FunctionTok{cat}\NormalTok{(}\StringTok{"}\SpecialCharTok{\textbackslash{}n}\StringTok{Highest predicted rentals:"}\NormalTok{, }\FunctionTok{names}\NormalTok{(}\FunctionTok{which.max}\NormalTok{(season\_predictions)), }\StringTok{"}\SpecialCharTok{\textbackslash{}n}\StringTok{"}\NormalTok{)}
\end{Highlighting}
\end{Shaded}

\begin{verbatim}
## 
## Highest predicted rentals: Winter.(Intercept)
\end{verbatim}

\begin{Shaded}
\begin{Highlighting}[]
\FunctionTok{cat}\NormalTok{(}\StringTok{"Lowest predicted rentals:"}\NormalTok{, }\FunctionTok{names}\NormalTok{(}\FunctionTok{which.min}\NormalTok{(season\_predictions)), }\StringTok{"}\SpecialCharTok{\textbackslash{}n}\StringTok{"}\NormalTok{)}
\end{Highlighting}
\end{Shaded}

\begin{verbatim}
## Lowest predicted rentals: Spring.(Intercept)
\end{verbatim}

\textbf{Answer:} According to the model, assuming all other variables
are held constant at their reference levels, \textbf{Summer} is
predicted to have the highest total daily bike rentals, while
\textbf{Winter} is predicted to have the lowest. This aligns with the
exploratory data analysis showing strong seasonal effects, with warm
seasons promoting higher bike rental demand and cold seasons reducing
demand.

Knit, \emph{commit, and push your changes to GitHub with an appropriate
commit message. Make sure to commit and push all changed files so that
your Git pane is cleared up afterwards and review the md document on
GitHub to make sure you're happy with the final state of your work.}

\end{document}
